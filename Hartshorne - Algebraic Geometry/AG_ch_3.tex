
	%Chapter III
	\chapter{Cohomology (코호몰로지)}
	
	이 장에서 우리는 위상공간에서의 가환군의 층의 코호몰로지의 일반적 개념을 정의할 것이고
	그 후 Noether 스킴에서의 연접 및 준연접층의 코호몰로지를 자세히 다룰 것이다.
	
	마지막 결과는 일반적으로 동일함에도 불구하고 코호몰로지를 도입하는 서로 다른 여러 방법들이 존재한다.
	때로는 복소다변수론에서도 사용되는 좋은 방법도 있다 - Gunning and Rossi [1]을 참조하라;
	추상적 대수기하학에 처음으로 코호몰로지를 도입한 Serre [3]는 \v Cech 코호몰로지를 사용했다;
	Grothendieck [1]의 유도 함자 접근법도 있다. 이들은 모두 각각의 중요성을 가진다.
	
	우리는 기본 정의를 대역적 단면 함자의 유도 함자로 취할 것이다. (\S 1, 2)
	이 정의는 가장 일반적이고 \S 7의 Serre 쌍대성의 증명과 같은 이론적인 문제에 가장 적합하다.
	그러나 이는 현실적으로 계산하는 것이 불가능하며 따라서 우리는 \S 4에서 \v Cech 코호몰로지를 도입하고
	이를 \S 5에서 사영공간 $\mb P^r$에서의 층 $\mc O(n)$들의 코호몰로지를 명시적으로 계산하기 위해 사용할 것이다.
	이러한 계산은 사영 대수다양체에 대한 이후의 여러 결과들의 기반이 된다.
	
	\v Cech 코호몰로지가 유도 함자 코호몰로지와 일치함을 증명하기 위해서는 아핀 스킴에서의 준연접층의 고차 코호몰로지가 0임을 알아야 한다.
	일반적인 아핀 스킴의 경우는 ([EGA III, \S 1]) 기술적으로 상당히 복잡하므로
	우리는 이를 \S 3에서 Noether 스킴의 경우에 대해서만 증명할 것이다.
	그러므로 우리는 코호몰로지를 수반하는 모든 정리에 Noether 전제조건을 포함하도록 제약될 것이다.
	
	코호몰로지의 응용으로 우리는 (I, \S 7)에서 $X$의 사영적 매장에 의존하도록 정의된 사영 대수다양체 $X$의 산술종수가
	코호몰로지 군 $H^i(X,\mc O_X)$를 통해 계산도리 수 있으며 따라서 내재적임을 보일 것이다. (Ex. 5.3)
	또한 우리는 산술종수가 정규 사영 대수다양체의 족에서 상수임을 보일 것이다. (9.13)
	
	다른 응용은 대수다양체의 쌍유리 이론에서 중요한 Zariski의 주요 정리(11.4)이다.
	
	이 장의 뒤쪽 부분(\S 8 - \S 12)은 스킴의 족에 할당되었다. i.e. 사상의 올에 관한 연구이다.
	특히 우리는 평탄 사상에서의 단면과 매끄러운 사상에서의 단면을 도입할 것이다.
	이들은 코호몰로지 없이도 다룰 수 있지만 평탄성은 코호몰로지를 이용하면 더 잘 이해할 수 있으므로 (9.9)
	이들을 삽입하기에 적절한 장소라 생각한다.
	
	
	%Section 1
	\section{Derived Functors (유도 함자)}
	
	이 장에서 우리는 호몰로지 대수의 기초적인 기법에 익숙하다 가정할 것이다.
	표기법과 용어가 자료마다 모두 다르므로 우리는 이 절에서 기초적인 정의 및 필요한 결과들을 (증명 없이) 모아놓을 것이다.
	세부사항은 다음 자료들에서 찾을 수 있다: Godement [1, esp. Ch. I, \S 1.1-1.8, 2.1-2.4, 5.1-5.3],
	Hilton and Stammbach [1, Ch. II, IV, IX], Grothendieck [1, Ch. II, \S 1, 2, 3],
	Cartan and Eilenberg [2, Ch. III, IV], Rotman [I, \S 6].
	
	
	%Definition
	\begin{definition}
	\tb{Abel 범주(abelian category)}는 다음을 만족시키는 범주 $\mf A$이다:
	각각의 $A,B\in\Ob\mf U$에 대하여 $\Hom(A,B)$가 가환군 구조를 가지며 사상의 합성이 선형적이다;
	유한 직접합이 존재한다; 모든 사상이 핵과 여핵을 가진다; 모든 단사 사상이 그 여핵의 핵이며 모든 전사 사상이 그 핵의 여핵이다;
	마지막으로 모든 사상이 전사 사상에 단사 사상을 합성한 것으로 분해될 수 있다. (Hilton and Stammbach [1, p. 78])
	\end{definition}
	
	다음은 모두 Abel 범주이다.
	
	
	%Example 1.0.1
	\begin{example}
	가환군의 범주 $\Ab$
	\end{example}
	
	
	%Example 1.0.2
	\begin{example}
	(단위 가환)환 $A$ 상에서의 모듈의 범주 $\Mod(A)$
	\end{example}
	
	
	%Example 1.0.3
	\begin{example}
	위상공간 $X$에서의 가환군의 층의 범주 $\Ab(X)$
	\end{example}
	
	
	%Example 1.0.4
	\begin{example}
	환 달린 공간 $(X,\mc O_X)$에서의 층의 범주 $\Mod(X)$
	\end{example}
	
	
	%Example 1.0.5
	\begin{example}
	스킴 $X$에서의 $\mc O_X$-모듈의 준연접층(II, 5.7)의 범주 $\Qco(X)$
	\end{example}
	
	
	%Example 1.0.6
	\begin{example}
	Noether 스킴 $X$에서의 $\mc O_X$-모듈의 연접층(II, 5.7)의 범주 $\Coh(X)$
	\end{example}
	
	
	%Example 1.0.7
	\begin{example}
	Noether 형식적 스킴 $\mf X$에서의 $\mc O_{\mf X}$-모듈의 연접층(II, 9.9)의 범주 $\Coh(\mf X)$
	\end{example}
	
	이 절의 나머지 부분에서 우리는 임의의 Abel 범주의 맥락에서 호몰로지 대수의 기초적인 결과들을 기술할 것이다.
	그러나 많은 책에서 이러한 결과들은 환 상에서의 모듈의 범주에 대해서만 증명되며 증명은 종종 `도표 추적(diagram chasing)'에 의해 수행된다:
	원소를 선택하고 도표 내에서 그 상과 원상들을 추적한다.
	도표 추적은 임의의 Abel 범주에서 사용 불가능하므로 성실한 독자들은 불안할 수 있다.
	이러한 문제를 해결하는 적어도 3가지 방법이 있다.
	(1) 원소를 전혀 언급하지 않고 모든 결과에 대하여 Abel 범주의 공리에서 시작하는 내재적 증명을 제시한다.
	이는 번잡하지만 불가능하지는 않다 - 예를 들어 Freyd [1]을 참조하라.
	(2) 우리가 사용하는 각각의 범주에서 (대부분은 위 목록에 속한 예시들이다) 도표 추적에 의한 증명을 수행할 수 있다는 사실을 인지한다.
	(3) `충만 매장 정리(full embedding theorem)' (Freyd [1, Ch. 7])을 받아들인다.
	이는 대략적으로 임의의 Abel 범주가 $\Ab$의 부분범주와 동치라고 기술한다.
	이는 (e.g. 도표 추적에 의해) $\Ab$에서 증명될 수 있는 임의의 범주론적 진술(e.g. 5-보조정리)이 임의의 Abel 범주에서도 성립함을 함의한다.\\
	
	이제 호몰로지 대수에 대한 복습을 시작하겠다. Abel 범주 $\mf A$에서의 \tb{복합체(complex)} $A^\cdot$은
	$i\in\Z$에 대한 대상 $A^i$들과 임의의 $i$에 대하여 $d^{i+1}\circ d^i=0$을 만족시키는 사상 $d^i:A^i\ra A^{i+1}$들의 족이다.
	만약 모든 대상 $A^i$들이 어떠한 범위에서만 명시된다면 (e.g. $i\ge 0$) 다른 모든 $i$에 대하여 $A^i=0$이라 설정한다.
	복합체의 \tb{사상(morphism)} $f:A^\cdot\ra B^\cdot$은 쌍대경계사상 $d^i$들과 교환 가능한 각각의 $i$에 대한 사상 $f^i:A^i\ra B^i$
	
	복합체 $A^\cdot$의 $i$번째 \tb{코호몰로지 대상(cohomology object)} $h^i(A^\cdot)$은 $\ket d^i/\im d^{i-1}$로 정의된다.
	만약 $f:A^\cdot\ra B^\cdot$가 복합체 사상이면 $f$는 자연스러운 사상 $h^i(f):h^i(A^\cdot)\ra h^i(B^\cdot)$를 유도한다.
	만약 $0\ra A^\cdot\ra B^\cdot\ra C^\cdot\ra 0$이 복합체의 짧은 완전열이면
	자연스러운 사상 $\de^i:h^i(C^\cdot)\ra h^{i+1}(A^\cdot)$이 존재하여 다음의 긴 완전열을 제공한다.
	%
	$$\cdots\ra h^i(A^\cdot)\ra h^i(B^\cdot)\ra h^i(C^\cdot)\sr{\de^i}\ra h^{i+1}(A^\cdot)\ra\cdots$$
	
	복합체의 두 사상 $f,g:A^\cdot\ra B^\cdot$가 \tb{호모토픽(homotopic)}하다는 것은 $f\sim g$로 표기하며
	($d^i$들과 교환 가능할 필요는 없는) 각각의 $i$에 대한 사상 $k^i:A^i\ra B^{i-1}$들의 족이 존재하여 $f-g=dk+kd$를 만족시키는 것이다.
	사상들의 족 $k=(k^i)$는 \tb{호모토피 작용소(homotopy operator)}라 불린다.
	만약 $f\sim g$이면 $f$와 $g$는 각각의 $i$에 대하여 코호몰로지 대상에서 \ti{동일한} 사상 $h^i(A^\cdot)\ra h^i(g^\cdot)$을 유도한다.
	
	Abel 범주 간의 공변 함자 $F:\mf A\ra\mf B$가 \tb{덧셈적(additive)}이라는 것의 정의는
	$\mf A$에서의 임의의 두 대상 $A,A'$에 대하여 유도된 함수 $\Hom(A,A')\ra\Hom(FA,FA')$이 가환군의 준동형사상인 것이다.
	$F$가 \tb{좌 완전(left exact)}이라는 것의 정의는 덧셈적이며 $\mf A$에서의 모든 짧은 완전열
	%
	$$0\ra A'\ra A\ra A''\ra 0$$
	%
	에 대하여 다음 열이 $\mf B$에서의 완전열인 것이다.
	%
	$$0\ra FA'\ra FA\ra FA''$$
	%
	좌측 대신 우측에 0을 쓸 수 있다면 $F$가 \tb{우 완전(right exact)}이라 한다. 만약 좌우 양쪽에서 완전하면 \tb{완전(exact)}이라 한다.
	중간 부분 $FA'\ra FA\ra FA''$이 완전하면 $F$가 \tb{중간에서 완전(exact in the middle)}이라 한다.
	
	반변 함자에 대해서도 유사하게 정의한다. 예를 들어 $F:\mf A\ra\mf B$가 \tb{좌 완전(left exact)}이라는 것의 정의는
	덧셈적이며 위에서와 같은 모든 짧은 완전열에 대하여 다음이 $\mf B$에서의 완전열인 것이다.
	%
	$$0\ra FA''\ra FA\ra FA'$$
	
	
	%Example 1.0.8
	\begin{example}
	만약 $\mf A$가 Abel 범주이며 $A$가 고정된 대상이면 함자 $B\ra\Hom(A,B)$는
	통상적으로 $\Hom(A,\cdot)$로 표기되며 $\mf A$에서 $\Ab$로의 공변 좌 완전 함자이다.
	함자 $\Hom(\cdot,A)$는 $\mf A$에서 $\Ab$로의 반변 좌 완전 함자이다.
	\end{example}
	
	다음으로 우리는 분해와 유도 함자를 다룰 것이다. $\mf A$의 대상 $I$가 \tb{단사(injective)}임은 함자 $\Hom(\cdot,I)$가 완전 함자인 것이다.
	$\mf A$의 대상 $A$의 \tb{단사 분해(injective resolution)}는
	차수 $i\ge 0$에서 정의된 복합체 $I^\cdot$와 사상 $A\ra I^0$로 다음을 만족시키는 것이다:
	각각의 $i$에 대하여 $I^i$가 $\mf A$에서의 단사 대상이며 다음의 열이 완전열이다.
	%
	$$0\ra A\sr\ep\ra I^0\ra I^1\ra\cdots$$
	
	만약 $\mf A$의 모든 대상이 $\mf A$의 단사 대상의 부분대상과 동형이면
	$\mf A$가 \tb{충분한 단사 대상을 가졌다(has enough injectives)}고 한다.
	만약 $\mf A$가 충분한 단사 대상을 가졌다면 모든 대상은 단사 분해를 가진다.
	특히 잘 알려진 보조정리는 임의의 두 단사 분해가 호모토피 동치라 기술한다.
	
	이제 $\mf A$가 충분한 단사 대상을 가지는 Abel 범주이며 $F:\mf A\ra\mf B$가 공변 좌 완전 함자라 하자.
	그 경우 $F$의 \tb{우 유도 함자(right derived functor)} $R^iF,F\ge 0$을 다음과 같이 구축한다:
	$\mf A$의 각각의 대상 $A$에 대하여 $A$의 단사 분해 $I^\cdot$를 하나씩 선택하자.
	그 경우 $R^iF(A)=h^i(F(I^\cdot))$로 정의한다.
	
	
	
	
	
	
	
	
	
	
	
	
	
	
	
	
	
	
	
	
	
	
	
	
	
	
	
	